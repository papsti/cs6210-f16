\section{Canonical forms}

Abstract linear algebra is about vector spaces and the operations on
them, independent of any specific choice of basis.  But while the
abstract view is useful, when we compute, we  are concrete, working with
the vector spaces $\bbR^n$ and $\bbC^n$ with a standard norm or inner
product structure.  A choice of basis (or choices of bases) links the
abstract view to a particular representation.  When working with an
abstract inner product space, we often would like to choose an
orthonormal basis, so that the inner product in the abstract space
corresponds to the standard dot product in $\bbR^n$ or $\bbC^n$.
Otherwise, the choice of basis may be arbitrary in principle ---
though, of course, some bases are particularly useful for revealing
the structure of the operation.

For any given class of linear algebraic operations, we have
\emph{equivalence classes} of matrices that represent the operation
under different choices of bases.  It is useful
to choose a distinguished representative for each of these
equivalence classes, corresponding to a choice of basis that
renders the structure of the operation particularly clear.
These distinguished representatives are known as \emph{canonical forms}.
Many of these equivalence relations have special names, as do many of
the canonical forms.

For spaces without and with inner product structure, the equivalence relations
and canonical forms associated with an operation on $\calV$ of dimension
$n$ and $\calW$ of dimension $n$ are shown in Figure~\ref{fig-canon}.
A major theme in the analysis of the Hermitian eigenvalue problem follows
from a pun: in the Hermitian case in an inner product space,
the equivalence relation for operators (unitary similarity) and for
quadratic forms (unitary congruence) are the same thing!.

% Requires pdflscape, afterpage, and capt-of
\afterpage{
\clearpage
\begin{landscape}
\centering
\begin{tabular}{l|l|l|l}
  Abstract & Concrete & Equivalence & Canonical Form \\ \hline \hline
  %%%%%%%%%%%%%%%%%%%%%%%%%%%%%%%%%%%%
  Linear map &
  $w = Av$ &
  $A \sim X^{-1} A Y$ &
  $\begin{bmatrix} I_k & 0 \\ 0 & 0 \end{bmatrix}$ \\
  $\mathcal{A} : \calV \rightarrow \calW$ &
  $A \in \bbC^{m \times n}$ &
  $X \in \bbC^{m \times m}, Y \in \bbC^{n \times n}$ invertible &
  $k = \operatorname{rank}(A)$ \\ \hline
  & &
  $A \sim U^* A V$ &
  $\Sigma = \ddiag(\sigma_1, \ldots, \sigma_{\min(m,n)})$ \\
  & &
  $U \in \bbC^{m \times m}, V \in \bbC^{n \times n}$ orthogonal &
  SVD \\ \hline \hline
  %%%%%%%%%%%%%%%%%%%%%%%%%%%%%%%%%%%%
  Operator &
  $v' = Av$ &
  $A \sim X^{-1} A X$ (similarity) &
  Jordan form \\
  $\mathcal{A} : \calV \rightarrow \calV$ &
  $A \in \bbC^{n \times n}$ &
  $X \in \bbC^{n \times m}$ invertible & \\ \hline
  & &
  $A \sim U^* A U$ (unitary similarity) &
  $T \in \bbC^{n \times n}$ upper triangular \\
  & &
  $U \in \bbC^{n \times n}$ unitary & Schur form \\ \hline \hline
  %%%%%%%%%%%%%%%%%%%%%%%%%%%%%%%%%%%%
  Quadratic form &
  $\phi = v^* Av$ &
  $A \sim X^* A X$ (congruence) &
  $\ddiag(I_{\nu_+}, 0_{\nu_0}, -I_{\nu_-})$ \\
  $\phi: \calV \times \calV \rightarrow \bbR$ &
  $A = A^* \in \bbC^{n \times n}$ &
  $X \in \bbC^{n \times m}$ invertible &
  $\nu = (\nu_+, \nu_-, \nu_0)$ = {\em inertia} of $A$ \\ \hline
  & &
  $A \sim U^* A U$ (unitary congruence) &
  $\ddiag(\lambda_1, \ldots, \lambda_n)$ \\
  & &
  $U \in \bbC^{n \times m}$ unitary & Eigendecomposition \\ \hline \hline
\end{tabular}
\captionof{table}{Complex canonical forms (without and with inner product structure)}
\label{fig-canon}
\clearpage
\end{landscape}
}
\newpage
