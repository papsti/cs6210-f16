\section{Sylvester's inertia theorem}

The inertia $\nu(A)$ is a triple consisting of the number of positive,
negative, and zero eigenvalues of $A$.  {\em Sylvester's inertia
  theorem} says that inertia is preserved under nonsingular {\em
  congruence} transformations, i.e. transformations of the form
\[
  M = V^* A V
\]
where $V$ is nonsingular (but not necessarily unitary).

Congruence transformations are significant because they are the natural
transformations for {\em quadratic forms} defined by symmetric matrices;
and the invariance of inertia under congruence says something about the
invariance of the shape of a quadratic form under a change of basis.
For example, if $A$ is a positive (negative) definite matrix, then the
quadratic form
\[
  \phi(x) = x^* A x
\]
defines a concave (convex) bowl; and $\phi(Vx) = x^* (V^* A V) x$ has
the same shape.

As with almost anything else related to the symmetric eigenvalue
problem, the minimax characterization is the key to proving
Sylvester's inertia theorem.  The key observation is that if
$M = V^* A V$ and $A$ has $k$ positive eigenvalues, then the minimax theorem
gives us a $k$-dimensional subspace $\calW_+$ on which $A$ is positive
definite (i.e. if $W$ is a basis, then $z^* (W^* A W) z > 0$ for any
nonzero $z$).  The matrix $M$ also has a $k$-dimensional space on
which it is positive definite, namely $V^{-1} \calW$.  Similarly, $M$
and $A$ both have $(n-k)$-dimensional spaces on which they are
negative semidefinite.  So the number of positive eigenvalues of $M$
is $k$, just as the number of positive eigenvalues of $A$ is $k$.
