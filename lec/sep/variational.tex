\section{Variational approaches}

% The Rayleigh quotient (standard or gen), weighted eigenvalue combo
% Implications for the SVD (two-sided)
% Constrained optimization and Lagrange multipliers
% Residual minimization

The Rayleigh quotient plays a central role in the theory of the
symmetric eigenvalue problem.  Recall that the Rayleigh quotient is
\[
  \rho_A(v) = \frac{v^* A v}{v^* v}.
\]
Substituting in $A = U \Lambda U^*$ and (without loss of generality)
assuming $w = U^* v$ is unit length, we have
\[
  \rho_A(v) = \sum_{i=1}^N \lambda_i |w_i|^2, \quad
  \mbox{ with } \sum_{i=1}^N |w_i|^2 = 1.
\]
That is, the Rayleigh quotient is a weighted average of the eigenvalues.
Maximizing or minimizing the Rayleigh quotient therefore yields the
largest and the smallest eigenvalues, respectively; more generally,
for a fixed $A$,
\[
  \delta \rho_A(v) = \frac{2}{\|v\|^2} \, \delta_v^* \left( A v - v \rho_A(v) \right),
\]
and so at a stationary $v$ (where all derivatives are zero),
we satisfy the eigenvalue equation
\[
  Av = v \rho(A).
\]
The eigenvalues are the stationary values of $\rho_A$; the eigenvectors
are stationary vectors.

The Rayleigh quotient is homogeneous of degree zero; that is, it is
invariant under scaling of the argument, so $\rho_A(v) = \rho_A(\tau v)$
for any $\tau \neq 0$.  Hence, rather than consider the problem of
finding stationary points of $\rho_A$ generally, we might restrict our
attention to the unit sphere.  That is, consider the Lagrangian function
\[
  L(v,\lambda) = v^* A v - \lambda (v^* v - 1);
\]
taking variations gives
\[
  \delta L = 2 \delta v^* (Av -\lambda v) - \delta \lambda (v^* v - 1)
\]
which is zero only if $Av = \lambda v$ and $v$ is normalized to unit
length.  In this formulation, the eigenvalue is identical to the
Lagrange multiplier that enforces the constraint.

The notion of a Rayleigh quotient generalizes to pencils.
If $M$ is Hermitian and positive definite, then
\[
  \rho_{A,M}(v) = \frac{v^* A v}{v^* M v}
\]
is a weighted average of generalized eigenvalues, and the stationary
vectors satisfy the generalized eigenvalue problem
\[
  Av = Mv \rho_{A,M}(v).
\]
We can also restrict to the ellipsoid $\|v\|_M^2 = 1$, i.e. consider
the stationary points of the Lagrangian
\[
  L(v,\lambda) = v^* A v - \lambda (v^* M v - 1),
\]
which again yields a generalized eigenvalue problem.

The analogous construction for the SVD is
\[
  \phi(u,v) = \frac{u^* A v}{\|u\| \|v\|}
\]
or, thinking in terms of a constrained optimization problem, 
\[
  L(u,v,\lambda,\mu) = u^* A v - \lambda (u^* u - 1) - \mu (v^* v-1).
\]
Taking variations gives
\[
  \delta L =
  \delta u^* (Av - 2\lambda u) + \delta v^* (A^* u-2\mu v) - \delta \lambda (u^*u - 1) - \delta \mu (v^* v - 1),
\]
and so $Av \propto u$ and $A^* u \propto v$.  Combining these observations
gives $A^* A v \propto v$, $A A^* u \propto u$, which we recognize as one
of the standard eigenvalue problem formulations for the SVD, with the squared
singular values as the constant of proportionality.
