\section{The slippery inverse}

The concept of the inverse of a matrix is generally more useful in
theory than in numerical practice.  We work with the inverse
{\em implicitly} all the time through solving linear systems via LU; but
we rarely form it {\em explicitly}, unless the inverse has some
special structure we want to study or to use.

If we did want to form $A^{-1}$ explicitly, the usual approach is to
compute $PA = LU$, then use that factorization to solve the systems
$Ax_k = e_k$, where $e_k$ is the $k$th column of the identity matrix
and $x_k$ is thus the $k$th column of the identity matrix.  As
discussed last time, forming the $LU$ factorization takes $n^3/3$
multiply-adds ($2n^3/3$ flops), and a pair of triangular solves takes
$n^2$ multiply-add operations.  Therefore, computing the inverse
explicitly via an LU factorization takes about $n^3$ multiply-add
operations, or roughly three times as much arithmetic as the original
LU factorization.  Furthermore, multiplying by an explicit inverse is
almost exactly the same amount of arithmetic work as a pair of
triangular solves.  So computing and using an explicit inverse is, on
balance, more expensive than simply solving linear systems using the
LU factorization.

To make matters worse, multiplying by the explicit inverse of a matrix
is {\em not} a backward stable algorithm.  Even if we could compute
$A^{-1}$ essentially exactly, only committing rounding errors when
storing the entries and when performing matrix-vector multiplication,
we would find $\fl(A^{-1} b) = (A^{-1} + F)b$, where $|F| \leq n
\macheps |A^{-1}|$.  But this corresponds to to a backward error of
roughly $-AFA$, which is potentially much larger than $\|A\|$.

In summary: you should get used to the idea that any time you see an
inverse in the description of a numerical method, it is probably
shorthand for ``solve a linear system here.''  Except in special
circumstances, forming and multiplying by an explicit inverse is both
slower and less numerically stable than solving a linear system by
Gaussian elimination.
