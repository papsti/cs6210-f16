\section{Tridiagonal systems}

Consider a symmetric positive definite tridiagonal system
\[
  A = \begin{bmatrix}
  \alpha_1 & \beta_1 \\
  \beta_1 & \alpha_2 & \beta_2 \\
  & \beta_2 & \alpha_3 & \beta_3 \\
  & & \ddots & \ddots & \ddots \\
  & & & \beta_{n-2} & \alpha_{n-1} & \beta_{n-1} \\
  & & & & \beta_{n-1} & \alpha_n
  \end{bmatrix}
\]
If we do one step of Cholesky factorization, the first column of
multipliers is nonzero only in the first two entries,
while the Schur complement is
\[
  S = \begin{bmatrix}
  \alpha_2-\beta_1^2/\alpha_1 & \beta_2 \\
  \beta_2 & \alpha_3 & \beta_3 \\
  & \ddots & \ddots & \ddots \\
  & & \beta_{n-2} & \alpha_{n-1} & \beta_{n-1} \\
  & & & \beta_{n-1} & \alpha_n
  \end{bmatrix}.
\]
That is, at the first step, $L$ and $S$ retain {\em exactly the same
nonzero structure} as the original tridiagonal $A$.  Cholesky
factorization on a tridiagonal therefore runs in $O(n)$ time.

More generally, unpivoted {\em band elimination} retains the structure
of the $A$ matrix in the $LU$ factors: if $A$ has lower and upper
bandwidths $p$ and $q$, then $L$ and $U$ have lower and upper
bandwidths $p$ and $q$, respectively.  With pivoting, the upper
bandwidth of $L$ can go up to $p+q$, and there are at most
$p+1$ nonzeros per column of $L$.

LAPACK has specialized LU routines for SPD and general nonsymmetric
tridiagonal and banded matrices.
