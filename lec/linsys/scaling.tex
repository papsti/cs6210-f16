\section{Scaling}

Suppose we wish to solve $Ax = b$ where $A$ is ill-conditioned.  Sometimes,
the ill-conditioning is artificial because we made a poor choice of units,
and it appears to be better conditioned if we write
\[
  D_1 A D_2 y = D_1 b,
\]
where $D_1$ and $D_2$ are diagonal scaling matrices.  If the original problem
was poorly scaled, we will likely find $\kappa(D_1 A D_2) \ll \kappa(A)$,
which may be great for Gaussian elimination.  But by scaling the matrix, we
are really changing the norms that we use to measure errors --- and that may
not be the right thing to do.

For physical problems, a good rule of thumb is to non-dimensionalize
before computing.  The non-dimensionalization will usually reveal a good
scaling that (one hopes) simultaneously is appropriate for measuring errors
and does not lead to artificially inflated condition numbers.
