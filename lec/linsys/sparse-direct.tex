\section{Sparse direct methods}

Suppose $A$ is a sparse matrix, and $PA = LU$.  Will $L$ and $U$ also
be sparse?  The answer depends in a somewhat complicated way on the
structure of the graph associated with the matrix $A$, the pivot
order, and the order in which variables are eliminated.  Except in
very special circumstances, there will generally be more nonzeros in
$L$ and $U$ than there are in $A$; these extra nonzeros are referred
to as {\em fill}.  There are two standard ideas for minimizing fill:
\begin{enumerate}
\item
  Apply a {\em fill-reducing ordering} to the variables; that is,
  use a factorization
  \[
    PAQ = LU,
  \]
  where $Q$ is a column permutation chosen to approximately minimize
  the fill in $L$ and $U$, and $P$ is the row permutation used for
  stability.

  The problem of finding an elimination order that minimizes fill is
  NP-hard, so it is hard to say that any ordering strategy is really
  optimal.  But there is canned software for some heuristic orderings
  that tend to work well in practice.  From a practical perspective,
  then, the important thing is to remember that a fill-reducing
  elimination order tends to be critical to using sparse Gaussian
  elimination in practice.
\item
  Relax the standard partial pivoting condition, choosing the row
  permutation $P$ to balance the desire for numerical stability
  against the desire to minimize fill.
\end{enumerate}

For the rest of this lecture, we will consider the simplified case of
{\em structurally} symmetric matrices and factorization without
pivoting (which you know from last week's guest lectures is stable for
diagonally dominany systems and positive definite systems).
