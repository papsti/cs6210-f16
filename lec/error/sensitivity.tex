
\section{Perturbing matrix problems}

To make the previous discussion concrete, suppose I want $y = Ax$, but
because of a small error in $A$ (due to measurement errors or roundoff
effects), I instead compute $\hat{y} = (A+E)x$ where $E$ is ``small.''
The expression for the {\em absolute} error is trivial:
\[
  \|\hat{y}-y\| = \|Ex\|.
\]
But I usually care more about the {\em relative error}.
\[
  \frac{\|\hat{y}-y\|}{\|y\|} = \frac{\|Ex\|}{\|y\|}.
\]
If we assume that $A$ is invertible and that we are using consistent
norms (which we will usually assume), then
\[
  \|Ex\| = \|E A^{-1} y\| \leq \|E\| \|A^{-1}\| \|y\|,
\]
which gives us
\[
  \frac{\|\hat{y}-y\|}{\|y\|} \leq \|A\| \|A^{-1}\|
  \frac{\|E\|}{\|A\|} = \kappa(A) \frac{\|E\|}{\|A\|}.
\]
That is, the relative error in the output is the relative error in the
input multiplied by the condition number
$\kappa(A) = \|A\| \|A^{-1}\|$.
Technically, this is the condition number for the problem of matrix
multiplication (or solving linear systems, as we will see) with
respect to a particular (consistent) norm; different problems have
different condition numbers.  Nonetheless, it is common to call this
``the'' condition number of $A$.

For some problems, we are given more control over the structure of the
error matrix $E$.  For example, we might suppose that $A$ is symmetric,
and ask whether we can get a tighter bound if in addition to assuming a
bound on $\|E\|$, we also assume $E$ is symmetric.  In this particular
case, the answer is ``no'' --- we have the same condition number either
way, at least for the 2-norm or Frobenius norm\footnote{This is left as
an exercise for the student}.  In other cases, assuming a structure to
the perturbation does indeed allow us to achieve tighter bounds.

As an example of a refined bound, we consider moving from condition
numbers based on small norm-wise perturbations to condition numbers
based on small {\em element-wise} perturbations.  Suppose $E$ is
elementwise small relative to $A$, i.e.~$|E| \leq \epsilon |A|$.
Suppose also that we are dealing with a norm such
that $\|X\| \leq \|~|X|~\|$,
as is true of all the norms we have seen so far.  Then
\[
  \frac{\|\hat{y}-y\|}{\|y\|} \leq
  \|EA^{-1}\| \leq \|~|A|~|A^{-1}|~\| \epsilon.
\]
The quantity $\kappa_{\mathrm{rel}}(A) = \|~|A|~|A^{-1}|~\|$ is
the {\em relative condition number};
it is closely related to the {\em Skeel condition number} which we will
see in our discussion of linear systems%
\footnote{The Skeel condition number involves the two factors
in the reverse order.}.
Unlike the standard condition number, the relative condition number is
invariant under column scaling of $A$;
that is $\kappa_{\mathrm{rel}}(AD) = \kappa_{\mathrm{rel}}(A)$
where $D$ is a nonsingular diagonal matrix.

What if, instead of perturbing $A$, we perturb $x$?  That is, if
$\hat{y} = A \hat{x}$ and $y = Ax$, what is the condition number
relating $\|\hat{y}-y\|/\|y\|$ to $\|\hat{x}-x\|/\|x\|$?
We note that
\[
  \|\hat{y}-y\| = \|A(\hat{x}-x)\| \leq \|A\| \|\hat{x}-x\|;
\]
and
\[
  \|x\| = \|A^{-1} y\| \leq \|A^{-1}\| \|y\|
  \quad \implies \quad
  \|y\| \geq \|A^{-1}\|^{-1} \|x\|.
\]
Put together, this implies
\[
  \frac{\|\hat{y}-y\|}{\|y\|} \leq \|A\| \|A^{-1}\|
  \frac{\|\hat{x}-x\|}{\|x\|}.
\]
The same condition number appears again!
