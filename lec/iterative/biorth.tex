\section{Bi-Lanczos}

So far, our focus has been on Krylov subspace methods that we can
explain via the Lanczos or Arnoldi decompositions.  The Lanczos-based
CG has many attractive properties, but it only works with symmetric
and positive definite matrices.  One can apply CG to a system of normal
equations --- the so-called CGNE method --- but this comes at the cost
of squaring the condition number.  There are also methods such as the
LSQR iteration that implicitly work with the normal equations, but use
an incrementally-computed version of the Golub-Kahan bi-diagonalization.
The Arnoldi-based GMRES iteration works for more general classes
of problems, and indeed it is the method of choice; but it comes at
a stiff penalty in terms of orthogonalization costs.

Are there alternative methods that use short recurrences (like CG) but
are appropriate for nonsymmetric matrices?  There are several, though
all have some drawbacks; the QMR and BiCG iterations may be the most
popular.  The key to the behavior of these methods comes from their
use of a different decomposition, the {\em bi-orthogonal Lanczos
factorization}:
\begin{align*}
  A Q_j &= Q_j T_j + \beta_{j+1} q_{j+1} e_j^* \\
  A^* P_j &= P_j T_j^* + \bar{\gamma}_{j+1} p_{j+1} e_j^* \\
  P_j^* Q_j &= I.
\end{align*}
Here, the bases $Q_j$ and $P_j$ span Krylov subspaces generated
by $A$ and $A^*$, respectively (which means these algorithms require
not only a function to apply $A$ to a vector, but also a function
to apply $A^*$).  The bases are not orthonormal, and indeed may become
rather ill-conditioned.  They {\em do} have a mutual orthogonality
relationship, though, namely $P_j^* Q_j = I$.

Details of the bi-orthogonal Lanczos factorization and related
iterative algorithms can be found in the references.  For the present,
we satisfy ourseves with a few observations:
\begin{itemize}
\item
  The GMRES iteration shows monotonic reduction in the preconditioned
  residual, even with restarting.  CG shows monotonic reduction in the
  error or residual when measured in appropriate norms.  The methods
  based on bi-orthogonal Lanczos, however, can show rather erratic
  convergence; errors decay in general, but they may exhibit
  intermediate local increases.  BiCG is generally more erratic than
  QMR.
\item
  Even in exact arithmetic, the subspace bases formed by bi-Lanczos
  may become rather ill-conditioned.
\item
  The bi-orthogonal iterations sometimes show {\em breakdown} behavior
  where the local approximation problem becomes singular.  This can be
  overcome using {\em lookahead} techniques, though it complicates the
  algorithm.
\end{itemize}
The relative simplicity of GMRES --- both in theory and in implementation ---
perhaps explains its relative popularity.  Nonetheless, these other methods
are worth knowing about.
