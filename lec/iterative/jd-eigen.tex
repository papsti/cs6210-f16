\subsection{Jacobi-Davidson}

The {\em Jacobi-Davidson} iteration is an alternative subspace-based
large-scale eigenvalue solver that does not use Krylov subspaces.
Instead, one builds a subspace via steps of an inexact Newton iteration
on the eigenvalue equation.  Given an approximate eigenpair $(\theta,u)$
where $\theta$ is the Rayleigh quotient, we seek a correction $s \perp u$
so that
\[
  A(u+s) = \lambda (u+s).
\]
Rewriting this in terms of $r = (A-\theta I) u$, we have for any
approximate $\tilde{\lambda}$ to the desired eigenvalue
\[
  (A-\tilde{\lambda} I) s = -r + (\lambda-\theta) u + (\lambda-\tilde{\lambda}) s.
\]
Using the desiderata that $u^* s = 0$ and the fact that $(I-uu^*) u = 0$,
we obtain the correction equation
\[
  (I-uu^*) (A-\tilde{\lambda} I)(I-uu^*) s = -r, \quad \mbox{where } s \perp u.
\]
The method proceeds by at each step solving the correction equation
approximately and extending the subspace by a new direction $s$.
One then seeks an approximate eigenpair from within the subspace.

In addition to a proper choice of subspaces, one needs a method to
extract approximate eigenvectors and eigenvalues.  This is particularly
important for approximating interior eigenvalues, as the standard
Rayleigh-Ritz approach may give poor results there.  One possible
method is to use the {\em refined Ritz} vector, which is obtained by
minimizing the residual over all candidate eigenvectors associated
with an approximate eigenvalue $\tilde{\lambda}$.  The refined Ritz
vector may then be plugged into the Rayleigh quotient to obtain a
new eigenvector.  Another method is the {\em harmonic Rayleigh-Ritz}
approach, which for eigenvalues near a target $\tau$ employs the
condition
\[
  (A-\tau I)^{-1} \tilde{u} - (\tilde{\theta}-\tau)^{-1} \tilde{u} \perp \mathcal{V}.
\]
We again usually use the Rayleigh quotient from the harmonic Ritz vector
rather than the harmonic Ritz value $\tilde{\theta}$.
