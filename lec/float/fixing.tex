\section{Finding and fixing floating point problems}

Floating point arithmetic is not the same as real arithmetic.  Even
simple properties like associativity or distributivity of addition and
multiplication only hold approximately.  Thus, some computations that
look fine in exact arithmetic can produce bad answers in floating
point.  What follows is a (very incomplete) list of some of the ways
in which programmers can go awry with careless floating point
programming.

\subsection{Cancellation}

If $\hat{x} = x(1+\delta_1)$ and $\hat{y} = y(1+\delta_2)$ are
floating point approximations to $x$ and $y$ that are very close, then
$\fl(\hat{x}-\hat{y})$ may be a poor approximation to $x-y$ due to
{\em cancellation}.  In some ways, the subtraction is blameless in
this tail: if $x$ and $y$ are close, then $\fl(\hat{x}-\hat{y}) =
\hat{x}-\hat{y}$, and the subtraction causes no additional rounding
error.  Rather, the problem is with the approximation error already
present in $\hat{x}$ and $\hat{y}$.

The standard example of loss of accuracy revealed through cancellation
is in the computation of the smaller root of a quadratic using the
quadratic formula, e.g.
\[
  x = 1-\sqrt{1-z}
\]
for $z$ small.  Fortunately, some algebraic manipulation gives an
equivalent formula that does not suffer cancellation:
\[
  x =
    \left( 1-\sqrt{1-z} \right)
    \left(\frac{1+\sqrt{1-z}}{1+\sqrt{1-z}}\right)
  =
    \frac{z}{1+\sqrt{1-z}}.
\]

\subsection{Sensitive subproblems}

We often solve problems by breaking them into simpler subproblems.
Unfortunately, it is easy to produce badly-conditioned subproblems
as steps to solving a well-conditioned problem.  As a simple (if
contrived) example, try running the following MATLAB code:

\lstinputlisting{code/sillysqrt.m}

In exact arithmetic, this should produce 2, but what does it produce
in floating point?  In fact, the first loop produces a correctly
rounded result, but the second loop represents the function
$x^{2^{60}}$, which has a condition number far greater than $10^{16}$
--- and so all accuracy is lost.

\subsection{Unstable recurrences}

One of my favorite examples of this problem is the recurrence relation
for computing the integrals
\[
  E_n = \int_{0}^1 x^n e^{x-1} \, dx.
\]
Integration by parts yields the recurrence
\begin{align*}
  E_0 &= 1-1/e \\
  E_n &= 1-nE_{n-1}, \quad n \geq 1.
\end{align*}
This looks benign enough at first glance: no single step of
this recurrence causes the error to explode.  But
each step amplifies the error somewhat, resulting in an exponential
growth in error\footnote{%
Part of the reason that I like this example is that one can
run the recurrence {\em backward} to get very good results,
based on the estimate $E_n \approx 1/(n+1)$ for $n$ large.
}.

\subsection{Undetected underflow}

In Bayesian statistics, one sometimes computes ratios of long
products.  These products may underflow individually, even when the
final ratio is not far from one.  In the best case, the products will
grow so tiny that they underflow to zero, and the user may notice an
infinity or NaN in the final result.  In the worst case, the
underflowed results will produce nonzero subnormal numbers with
unexpectedly poor relative accuracy, and the final result will be
wildly inaccurate with no warning except for the (often ignored)
underflow flag.

\subsection{Bad branches}

A NaN result is often a blessing in disguise: if you see an
unexpected NaN, at least you {\em know} something has gone wrong!
But all comparisons involving NaN are false,  and so
when a floating point result is used to compute a branch condition
and an unexpected NaN appears, the result can wreak havoc.
As an example, try out the following code in MATLAB.

\lstinputlisting{code/testnan.m}
