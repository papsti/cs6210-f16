\section{The Bauer-Fike theorem}

We now apply Gershgorin theory together with a carefully chosen
similarity to prove a bound on the eigenvalues of $A+F$ where $F$ is a
finite perturbation.  This will lead us to the {\em Bauer-Fike} theorem.

The basic idea is as follows.  Suppose that $A$ is a diagonalizable matrix, so
that there is a complete basis of column eigenvectors $V$ such that
\[
  V^{-1} A V = \Lambda.
\]
Then we $A+F$ has the same eigenvalues as
\[
  V^{-1} (A+F) V = \Lambda + V^{-1} F V = \Lambda + \tilde{F}.
\]
Now, consider the Gershgorin disks for $\Lambda + \tilde{F}$.
The crude bound (\ref{disk-bound-2}) tells us
that all the eigenvalues live in the regions
\[
  \bigcup_j \calB_{\|\tilde{F}\|_1}(\lambda_j) \; \subseteq \;
  \bigcup_j \calB_{\kappa_1(V) \|F\|_1}(\lambda_j).
\]
This bound really is crude, though; it gives us disks of the
same radius around all the eigenvalues $\lambda_j$ of $A$,
regardless of the conditioning of those eigenvalues.  Let's
see if we can do better with the sharper bound (\ref{disk-bound-1}).

To use (\ref{disk-bound-1}), we need to bound
the absolute column sums of $\tilde{F}$.  Let $e$ represent
the vector of all ones, and let $e_j$ be the $j$th column of
the identity matrix; then the $j$th absolute column sums of $\tilde{F}$
is $\phi_j \equiv e^T |\tilde{F}| e_j$, which we can bound
as $\phi_j \leq e^T |V^{-1}| |F| |V| e_j$.  Now, note that we
are free to choose the normalization of the eigenvector $V$;
let us choose the normalization so that each row of $W^* = V^{-1}$.
Recall that we defined the angle $\theta_j$ by
\[
  \cos(\theta_j) = \frac{|w_j^* v_j|}{\|w_j\|_2 \|v_j\|_2},
\]
where $w_j$ and $v_j$ are the $j$th row and column eigenvectors;
so if we choose $\|w_j\|_2 = 1$ and $w_j^* v_j = 1$ (so $W^* = V^{-1}$),
we must have $\|v_j\|_2 = \sec(\theta_j)$.  Therefore,
$\||V| e_j\|_2 = \sec(\theta_j)$.  Now, note that $e^T |V^{-1}|$ is
a sum of $n$ rows of Euclidean length 1, so $\|e^T |V^{-1}|\|_2 \leq n$.
Thus, we have
\[
  \phi_j \leq n \|F\|_2 \sec(\theta_j).
\]
Putting this bound on the columns of $\tilde{F}$ together with
(\ref{disk-bound-1}), we have the Bauer-Fike theorem.

\begin{theorem}
  Suppose $A \in \bbC^{n \times n}$ is diagonalizable with
  eigenvalues $\lambda_1, \ldots, \lambda_n$.
  Then all the eigenvalues of $A+F$ are in the region
  \[
    \bigcup_j \calB_{n \|F\|_2 \sec(\theta_j)}(\lambda_j),
  \]
  where $\theta_j$ is the acute angle between the row and column eigenvectors
  for $\lambda_j$, and any connected component $\calG$ of this region that
  contains exactly $m$ eigenvalues of $A$ will also contain exactly $m$
  eigenvalues of $A+F$.
\end{theorem}
