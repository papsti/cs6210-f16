\section{Spectral transformation and shift-invert}

Suppose again that $A$ is diagonalizable with $A = V \Lambda V^{-1}$.
The power iteration relies on the identity
\[
  A^k = V \Lambda^k V^{-1}.
\]
Now, suppose that $f(z)$ is any function that is defined locally by a
convergent power series.  Then as long as the eigenvalues are within
the radius of convergence, we can define $f(A)$ via the same power series,
and
\[
  f(A) = V f(\Lambda) V^{-1}
\]
where $f(\Lambda) = \operatorname{diag}(f(\lambda_1), f(\lambda_2),
\ldots, f(\lambda_n))$.  So the spectrum of $f(A)$ is the image of
the spectrum of $A$ under the mapping $f$, a fact known as the
{\em spectral mapping theorem}.

As a particular instance, consider the function $f(z) = (z-\sigma)^{-1}$.
This gives us
\[
  (A-\sigma I)^{-1} = V (\Lambda - \sigma I)^{-1} V^{-1},
\]
and so if we run power iteration on $(A-\sigma I)^{-1}$, we will converge
to the eigenvector corresponding to the eigenvalue $\lambda_j$ for
which $(\lambda_j-\sigma)^{-1}$ is maximal --- that is, we find the eigenvalue
closest to $\sigma$ in the complex plane.  Running the power method on
$(A-\sigma I)^{-1}$ is sometimes called the shift-invert power method.
