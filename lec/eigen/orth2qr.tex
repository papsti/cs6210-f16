\section{Orthogonal iteration to QR}

The focus of orthogonal iteration is the orthogonal (or unitary)
factor in the Schur form.  The upper triangular Schur factor
fades into the background, which is a pity; this is, after all, where we
learn the eigenvalues.  But through a bit of algebraic trickery, it turns
out that we can re-invent orthogonal iteration as an iteration for
the upper triangular factor.  The steps for this transformation are as
follows:
\begin{enumerate}
\item
  The orthogonal iteration $\uQ^{(k+1)} R^{(k)} = A \uQ^{(k)}$ is a
  generalization of the power method.  In fact, the first column of
  this iteration is {\em exactly} the power iteration.  In general,
  the first $p$ columns of $\uQ^{(k)}$ are converging to an orthonormal
  basis for a $p$-dimensional invariant subspace associated with the
  $p$ eigenvalues of $A$ with largest modulus (assuming that there
  aren't several eigenvalues with the same modulus to make this
  ambiguous).
\item
  If all the eigenvalues have different modulus, orthogonal iteration
  ultimately converges to the orthogonal factor in a Schur form
  \[
    AU = UT
  \]
  What about the $T$ factor?  Note that $T = U^* A U$, so a natural
  approximation to $T$ at step $k$ would be
  \[
    A^{(k)} = (\uQ^{(k)})^* A \uQ^{(k)},
  \]
  and from the definition of the subspace iteration, we have
  \[
    A^{(k)} = (\uQ^{(k)})^* \uQ^{(k+1)} R^{(k)} = Q^{(k)} R^{(k)},
  \]
  where $Q^{(k)} \equiv (\uQ^{(k)})^* \uQ^{(k+1)}$ is unitary.
\item
  Note that
  \[
    A^{(k+1)}
    = (\uQ^{(k+1)})^* A^{(k)} \uQ^{(k+1)}
    = (Q^{(k)})^* A^{(k)} Q^{(k)}
    = R^{(k)} Q^{(k)}.
  \]
  Thus, we can go from $A^{(k)}$ to $A^{(k+1)}$ directly without
  the orthogonal factors from subspace iteration, simply by computing
  \begin{align*}
    A^{(k)} &= Q^{(k)} R^{(k)} \\
    A^{(k+1)} &= R^{(k)} Q^{(k)}.
  \end{align*}
  This is the {\em QR iteration}.
\end{enumerate}

Under some restrictions on $A$, the matrices $A^{(k)}$ will ultimately
converge to the triangular Schur factor of $A$.  But we have two problems:
\begin{enumerate}
\item
  Each step of the QR iteration requires a QR factorization, which is
  an $O(n^3)$ operation.  This is rather expensive, and even in the happy
  case where we might be able to get each eigenvalue with a constant
  number of steps, $O(n)$ total steps at a cost of $O(n^3)$ each gives
  us an $O(n^4)$ algorithm.  Given that everything else we have done
  so far costs only $O(n^3)$, an $O(n^4)$ cost for eigenvalue computation
  seems excessive.
\item
  Like the power iteration upon which it is based, the basic iteration
  converges linearly, and the rate of convergence is related to the
  ratios of the moduli of eigenvalues.  Convergence is slow when there
  are eigenvalues of nearly the same modulus, and nonexistent when
  there are eigenvalues with the same modulus.
\end{enumerate}
We describe next how to overcome these two difficulties.
