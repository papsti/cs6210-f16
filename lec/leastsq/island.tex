\section{A cautionary tale}

You have been dropped on a desert island with a laptop with a magic
battery of infinite life, a MATLAB license, and a complete lack of
knowledge of basic geometry.  In particular, while you know about
least squares fitting, you have forgotten how to compute the perimeter
of a square.  You vaguely feel that it ought to be related to the
perimeter or side length, though, so you set up the following model:
\[
  \mbox{perimeter} = \alpha \cdot \mbox{side length} + \beta \cdot \mbox{diagonal}.
\]
After measuring several squares, you set up a least squares system
$Ax = b$; with your real eyes, you know that this must look
like
\[
  A = \begin{bmatrix} s & \sqrt{2} s \end{bmatrix}, \quad
  b = 4 s
\]
where $s$ is a vector of side lengths.  The normal equations are
therefore
\[
A^T A = \|s\|^2
\begin{bmatrix} 1 & \sqrt{2} \\ \sqrt{2} & 2 \end{bmatrix}, \quad
A^T b = \|s\|^2
\begin{bmatrix} 4 \\ 4 \sqrt{2} \end{bmatrix}.
\]
This system does have a solution; the problem is that it has far more
than one.  The equations are singular, but consistent.
We have no data that would lead us to prefer to write
$p = 4s$ or $p = 2 \sqrt{2} d$ or something in between.
The fitting problem is {\em ill-posed}.

We deliberately started with an extreme case, but some ill-posedness
is common in least squares problems.  As a more natural example,
suppose that we measure the height, waist girth, chest girth, and
weight of a large number of people, and try to use these factors to
predict some other factor such as proclivity to heart disease.  Naive
linear regression -- or any other naively applied statistical
estimation technique -- is likely to run into trouble, as the height,
weight, and girth measurements are highly correlated.  It is not that
we cannot fit a good linear model; rather, we have too many models
that are each almost as good as the others at fitting the data!  We
need a way to choose between these models, and this is the point of
{\em regularization}.
