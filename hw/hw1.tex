\documentclass[12pt, leqno]{article} %% use to set typesize
\include{common}

\begin{document}

\hdr{HW 1}

You may (and should) talk about problems with each other and with me,
providing attribution for any good ideas you might get.  Your final
write-up should be your own.

\paragraph*{1: A problem of performance}
\matlab\ supports both sparse and dense matrix data structures,
and they have different performance characteristics.  For a variety
of square matrices of size $n$ and sparsity level $s$ (where $s$ is
the fraction of entries that are nonzero), compare the speed of dense
and sparse matrix-vector multiply.  You may use {\tt As = sparse(A)}
to make a sparse version of a dense matrix $A$.  What do you observe
about the relative performance of these options?

{\em Note:} Your performance will vary depending on your machine and
\matlab\ version.  Feel free to write these experiments using
Octave, Python, or Julia if you prefer.

\paragraph*{2: Toeplitz and transforms}
Consider a circulant matrix $C$ whose first column is $c$, i.e.
\[
  C =
  \begin{bmatrix}
  c_1 & c_n & c_{n-1} & \ldots & c_2 \\
  c_2 & c_1 & c_{n} & \ldots & c_3 \\
  \vdots & & & & \vdots \\
  c_n & c_{n-1} & c_{n-2} & \ldots & c_1 \\
  \end{bmatrix}.
\]
The following \matlab\ code computes a matrix-vector product with $C$
in $O(n \log n)$ time:
\begin{lstlisting}
  y = ifft(fft(c) .* fft(x));  % Equivalent to y = C*x
\end{lstlisting}
Using the circulant multiply as a building block and following the
approach sketched in the notes from the first week, write an $O(n \log n)$
multiply for the Toeplitz matrix $T$ with first row $r$ and first column $c$,
i.e.~a fast version of
\begin{lstlisting}
  y = toeplitz(c, r)*x;
\end{lstlisting}

\paragraph*{3: Low-rank limbo}
Suppose $A = xy^T$ where $x \in \bbR^m$ and $y \in \bbR^n$.  Show that
$\|A\|_2 = \|A\|_F = \|x\|_2 \|y\|_2$,
$\|A\|_1 = \|x\|_1 \|y\|_\infty$, and
$\|A\|_\infty = \|x\|_\infty \|y\|_1$.

\paragraph*{4: Noodling with Neumann}
Suppose that if $A \in \bbR^{n \times n}$ satisfies
$|A| < M$ (interpreted elementwise) for some $M \in \bbR^{n \times n}$.
Show that if $\|M\| < 1$ in some consistent norm, then
both $(I-A)^{-1}$ and $(I-M)^{-1}$ exist and
$|(I-A)^{-1}| \leq (I-M)^{-1}$.

\paragraph*{5: Simple substitutions}
Consider the code for forward substitution to solve the linear system
$Lx = b$ where $L$ is lower triangular:
\begin{lstlisting}
  for i = 1:n
    x(i) = ( b(i)-L(i,1:i-1)*x(1:i-1) )/L(i,i);
  end
\end{lstlisting}
Show that the vector $\hat{x}$ computed in floating point arithmetic
satisfies
\[
  (L+\delta L) \hat{x} = b
\]
where $|\delta l_{ij}| \leq n \macheps |l_{ij}| + O(\macheps^2)$.

\end{document}
